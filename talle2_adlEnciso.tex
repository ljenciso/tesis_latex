\documentclass[12pt, A4]{article}
\usepackage[utf8]{inputenc} % Para las tildes
\usepackage[spanish, es-tabla]{babel}
\usepackage{graphicx}
\usepackage{booktabs} % Better horizontal rules in tables
\usepackage{float}
\usepackage{amsmath}
\usepackage{palatino}
\usepackage{array} 
\usepackage{caption}
\usepackage{blindtext} 
\usepackage[table,xcdraw]{xcolor}
\usepackage{geometry}
\usepackage{hyperref}
\geometry{margin=1in}

%opening
\title{Taller 2 de análisis de datos longitudinales}
\author{Leonardo José Enciso Olivera}

\begin{document}

\maketitle

\section{Escenario propuesto}
Se recolectó información de 124 pacientes con diagnóstico de algún trastorno de salud mental. Los datos de las variables fueron recogidos en cuatro periodos de tiempo diferentes. El promedio de edad para la población fue de 29.23 años. De los pacientes incluidos, 91 (73.38\%) eran mujeres.

\section{Pregunta 1: Evalúe la relación entre la atención completa (desenlace) y el resto de variables utilizando un modelo GEE.}

Se inicia con el ordenamiento de los datos por identificador y tiempo. Para la construcción del modelo se toman en cuenta las siguientes características: 

\begin{itemize}
	\item El desenlace es dicótomo [0,1]
	\item La familia de distribución para el desenlace es binomial 
	\item La función de enlace que permite que la variable dependiente sea expresada como un vector de parámetros estimados ($\beta$) es logit
	\item Se evalúa la correlación entre las variables para definir la estructura apropiada de la matríz de trabajo. Se decide utilizar una forma no estructurada. 
\end{itemize}

\begin{table}[H]
	\centering
	\caption{Matriz de correlación de la variable de desenlace}
	\begin{tabular}{rrrrr}
		\hline
		& C.Completa T0 &  C.Completa T1 & C.Completa T2 & C.Completa T3 \\ 
		\hline
		C.Completa T0 & 1.00 & 0.17 & -0.22 & -0.05 \\ 
		C.Completa T1 & 0.17 & 1.00 & -0.01 & 0.08 \\ 
		C.Completa T2 & -0.22 & -0.01 & 1.00 & 0.17 \\ 
		C.Completa T3 & -0.05 & 0.08 & 0.17 & 1.00 \\ 
		\hline
	\end{tabular}
\end{table}

\subsection{Modelo completo}

Se inicia construyendo un modelo completo en donde se incluyen las variables edad, sexo y medicamentos. Las estimaciones del modelo se presentan en la tabla 2. 

\begin{table}[H]
	\caption{Estimaciones del modelo GEE}
	\label{tab:my-table}
	\begin{tabular}{llllll}
		& \multicolumn{1}{c}{\textbf{Coeficiente}} & \multicolumn{1}{c}{\textbf{Naive S.E}} & \multicolumn{1}{c}{\textbf{Naive Z}} & \multicolumn{1}{c}{\textbf{Robust S.E}} & \multicolumn{1}{c}{\textbf{Robust Z}} \\
		\textbf{(Intercept)} & -1,13048249 & 0,293604091 & -3,8503635 & 0,30434478 & -3,7144797 \\
		\textbf{edad} & 0,03526853 & 0,009654405 & 3,6531028 & 0,01036747 & 3,4018449 \\
		\textbf{sexo} & -0,0497991 & 0,151864677 & -0,3279176 & 0,14971089 & -0,3326351 \\
		\textbf{medicamentos} & 0,42908427 & 0,17740441 & 2,4186787 & 0,1852848 & 2,3158093
	\end{tabular}
\end{table}

Se realiza el cálculo de las razones de odds para cada variable con sus intervalos de confianza asociados. Se obtiene lo siguiente: 

\begin{table}[H]
	\centering
	\caption{Razones de Odds para las variables del modelo}
	\begin{tabular}{rrrr}
		\hline
		& OR & LI & LS \\ 
		\hline
		(Intercept) & 0.32 & 0.18 & 0.59 \\ 
		edad & 1.04 & 1.02 & 1.06 \\ 
		sexo & 0.95 & 0.71 & 1.28 \\ 
		medicamentos & 1.54 & 1.07 & 2.21 \\ 
		\hline
	\end{tabular}
\end{table}
En este modelo, cada incremento de un año de la edad y el consumo de los medicamentos durante el trimestre se asociaron con una mayor probabilidad de realizar las consultas completas. Para evaluar la hipótesis  de que $H_o=\beta Edad = \beta Medicamento = \beta Sexo = 0$, se realiza el cálculo del estadístico de Wald. Se encuentra que el valor del estadístico es de 17.63. Los grados de libertad corresponden al número de parámetros estimados (3) para una distribución $\chi^2$ siendo el valor p asociado de $0.0006$, rechazando la hipótesis nula e indicando que al menos uno de los beta es diferente de cero. La variable edad y la variable medicamentos tuvieron significancia estadísticas. Se evalua el comportamiento del modelo modificando la estructura de la matriz de trabajo de tipo intercambiable. 

\subsection{Modelo con cambio en la matriz de trabajo}
Se evalua el comportamiento del modelo modificando la estructura de la matriz de trabajo de tipo intercambiable. Los resultados se presentan en la tabla a continuación. 

\begin{table}[H]
	\caption{Modelo GEE con matriz de trabajo intercambiable}
	\label{tab:my-table}
	\begin{tabular}{@{}lccccc@{}}
		\toprule
		& \multicolumn{1}{l}{\textbf{Coeficiente}} & \multicolumn{1}{l}{\textbf{S.E Naive}} & \multicolumn{1}{l}{\textbf{Z Naive}} & \multicolumn{1}{l}{\textbf{S.E Robusto}} & \multicolumn{1}{l}{\textbf{Z Robusto}} \\ \midrule
		\textbf{(Intercept)} & -1,09331594 & 0,31323545 & -3,490397 & 0,31711469 & -3,4476988 \\
		\textbf{edad} & 0,04051539 & 0,01036883 & 3,907423 & 0,01064951 & 3,8044361 \\
		\textbf{sexo} & -0,04216095 & 0,16197932 & -0,260286 & 0,15750304 & -0,2676834 \\
		\textbf{medicamentos} & 0,17625957 & 0,20606454 & 0,855361 & 0,21150856 & 0,8333449 \\ \bottomrule
	\end{tabular}
\end{table}

Los valores de OR para este modelo con sus intervalos de confianza son: 

\begin{table}[H]
	\centering
	\begin{tabular}{rrrr}
		\hline
		& OR2 & LIr & LSr \\ 
		\hline
		(Intercept) & 0.34 & 0.18 & 0.62 \\ 
		edad & 1.04 & 1.02 & 1.06 \\ 
		sexo & 0.96 & 0.70 & 1.31 \\ 
		medicamentos & 1.19 & 0.79 & 1.81 \\ 
		\hline
	\end{tabular}
\end{table}

Se presenta la comparación de los coeficientes exponenciados (correspondientes a las razones de odds) de los dos modelos. 

\begin{table}[H]
	\centering
	\caption{Comparación de los OR de los dos modelos}
	\begin{tabular}{rrrrr}
		\hline
		& (Intercept) & edad & sexo & medicamentos \\ 
		\hline
		No estructurada & 0.32 & 1.04 & 0.95 & 1.54 \\ 
		Intercambiable & 0.34 & 1.04 & 0.96 & 1.19 \\ 
		\hline
	\end{tabular}
\end{table}

No se identifican mayores diferencias entre los resultados. Para realizar la comparación entre los modelos con diferentes estructuras de matriz de correlación, se utiliza el criterio de información de quasiverosimilitud de Pan (QIC). Los resultados se presentan a continuación. 

\begin{table}[H]
	\centering
	\caption{QIC para los dos modelos}
	\begin{tabular}{rr}
		\hline
		& QIC \\ 
		\hline
		No estructurado & 679.47 \\ 
		Intercambiable & 680.21 \\ 
		\hline
	\end{tabular}
\end{table}

El modelo con la matriz de trabajo no estructurada tiene un valor inferior, lo que sugiere que es el más adecuado para los datos. Se concluye entonces que en este modelo la edad y el consumo de medicamentos durante el trimestre se asociación con una mayor probabilidad de realizar las consultas completas de seguimiento. El sexo no tuvo significancia estadística en el modelo y debería considerarse su exclusión.

\section{Pregunta 2: Evalúe la relación entre número de consultas y el resto de variables utilizando un modelo GEE.}

Para la construcción del modelo se hacen las siguientes consideraciones: 

\begin{itemize}
	\item El desenlace es un desenlace de conteo y corresponde al número de consultas por cada sujeto en el tiempo y por tanto no son independientes. 
	\item La familia de distribución para el desenlace es Poisson 
	\item La función de enlace que permite que la variable dependiente sea expresada como un vector de parámetros estimados ($\beta$) es log
	\item Se selecciona inicialmente una matriz de trabajo no estructurada. Se presenta la matriz de correlación entre los datos en la tabla a continuación. 
\end{itemize}

\begin{table}[H]
	\caption{Matriz de correlación entre el número de consultas en el tiempo}
	\label{tab:my-table}
	\centering
	\begin{tabular}{lcccc}
		& \textbf{Consulta 0} & \textbf{Consulta 1} & \textbf{Consulta 2} & \textbf{Consulta 3} \\
		\textbf{Consulta 0} & 1 & 0,159024745 & -0,08407844 & -0,09360167 \\
		\textbf{Consulta 1} & 0,15902474 & 1 & 0,000869564 & 0,06734044 \\
		\textbf{Consulta 2} & -0,08407844 & 0,000869564 & 1 & 0,26521073 \\
		\textbf{Consulta 3} & -0,09360167 & 0,067340445 & 0,265210726 & 1
	\end{tabular}
\end{table}

\subsection{Modelo inicial con matriz de trabajo no estructurada}
Se inicia con un modelo que utiliza una matriz de trabajo no estructurada. Los resultados obtenidos se presentan en la tabla a continuación. 

\begin{table}[H]
	\caption{Modelo GEE para desenlace consultas}
	\label{tab:my-table}
	\begin{tabular}{@{}lccccc@{}}
		\toprule
		& \textbf{Coeficientes} & \textbf{S.E Naive} & \textbf{z Naive} & \textbf{S.E Robusto} & \textbf{z robusto} \\ \midrule
		\textbf{(Intercept)} & 1,66277061 & 0,047131437 & 35,279438 & 0,092537943 & 17,9685279 \\
		\textbf{edad} & 0,007495566 & 0,001533011 & 4,889442 & 0,002984192 & 2,5117573 \\
		\textbf{sexo} & -0,059945081 & 0,024878519 & -2,409512 & 0,048264835 & -1,2420032 \\
		\textbf{medicamentos} & 0,034211619 & 0,028001961 & 1,221758 & 0,051209228 & 0,6680753 \\ \bottomrule
	\end{tabular}
\end{table}

Los coeficientes indican el cambio en el log del número de consultas por cada cambio de unidad en la variable correspondiente dejando el resto igual. En este caso la edad en años, el sexo masculino (codificado 1) y el consumo de medicamentos. El único valor p  estadísticamente significativo es el de la variable edad (p valor = 0.012). El estadístico de Wald, considerando los tres parámetros estimados por el modelo es de 10.03 con tres grados de libertad, lo cual se asocia con un valor p ($\chi^2$) de 0.0183, indicando que al menos un coeficiente es diferente de cero.

La exponenciación de los coeficientes produce las razones de las tasas de incidencia del desenlace número de consultas por cambio de unidad de la covariable dejando el resto igual. Los resultados se presentan a continuación. 

\begin{table}[H]
	\centering
	\caption{IRR para las variables incluidas}
	\begin{tabular}{rrrr}
		\hline
		& IRR & LI\_IRR & LS\_IRR \\ 
		\hline
		(Intercept) & 5.27 & 4.40 & 6.32 \\ 
		edad & 1.01 & 1.00 & 1.01 \\ 
		sexo & 0.94 & 0.86 & 1.04 \\ 
		medicamentos & 1.03 & 0.94 & 1.14 \\ 
		\hline
	\end{tabular}
\end{table}

Los IRR para todas las variables son cercanos a 1 y en ninguna de las variables se obtienen valores estadísticamente significativos.

\subsection{Modelo con matriz de trabajo intercambiable}

Se realiza entonces la construcción de un modelo modificando la estructura de la matriz de trabajo a una estructura intercambiable y se realiza la comparación de los modelos utilizando el criterio de información de quasiverosimilitud de Pan (QIC). Se presenta inicialmente el resultado del modelo. 

\begin{table}[H]
	\caption{Modelo GEE con matriz de trabajo intercambiable}
	\label{tab:my-table}
	\begin{tabular}{@{}llllll@{}}
		\toprule
		& \multicolumn{1}{c}{\textbf{Coeficientes}} & \multicolumn{1}{c}{\textbf{S.E Naive}} & \multicolumn{1}{c}{\textbf{z Naive}} & \multicolumn{1}{c}{\textbf{S.E Robusto}} & \multicolumn{1}{c}{\textbf{z Robusto}} \\ \midrule
		\textbf{(Intercept)} & 1,75499371 & 0,056152279 & 31,2541845 & 0,07215903 & 24,3211945 \\
		\textbf{edad} & 0,00694705 & 0,001832611 & 3,7907945 & 0,0024882 & 2,7919982 \\
		\textbf{sexo} & -0,0599592 & 0,029675976 & -2,0204627 & 0,04270363 & -1,4040773 \\
		\textbf{medicamentos} & 0,01623777 & 0,037638814 & 0,4314103 & 0,05073164 & 0,3200719 \\ \bottomrule
	\end{tabular}
\end{table}

Se evalúan los coeficientes, encontrando que nuevamente el único cuyo valor z robusto se relaciona con un p-valor con significancia estadística es el de la variable edad (p-valor: 0.005). El estadístico de Wald fue de 11.33. Con tres grados de libertad se obtiene un valor p asociado de 0.01, rechanzando la hipótesis nula de que los coeficientes son iguales a cero. Se realiza entonces la exponenciación de los coeficientes para obtener las IRR. Los resultados se presentan a continuación. 

\begin{table}[H]
	\centering
	\begin{tabular}{rrrr}
		\hline
		& IRR Modelo 2 & LI & LS  \\ 
		\hline
		(Intercept) & 5.78 & 4.82 & 6.93 \\ 
		edad & 1.01 & 1.00 & 1.01 \\ 
		sexo & 0.94 & 0.86 & 1.04 \\ 
		medicamentos & 1.02 & 0.92 & 1.12 \\ 
		\hline
	\end{tabular}
\end{table}

De manera idéntica al modelo anterior, solamente fue significativo el valor de la IRR para la variable edad. Se realiza entonces la comparación entre los dos modelos utilizando el QIC. Se obtienen los siguientes resultados. 

\begin{table}[H]
	\centering
	\begin{tabular}{rr}
		\hline
		& QIC \\ 
		\hline
		No estructurado & -6594.73 \\ 
		Intercambiable & -6597.27 \\ 
		\hline
	\end{tabular}
\end{table}

La matríz de trabajo no estructurada es la que se considera más apropiada para los datos. El número de consultas en el tiempo 

\section{Pregunta 3. Evalúe la relación entre la atención completa y el resto de variables utilizando un modelo de efectos mixtos.}

Para evaluar la relación entre el tiempo y el desenlace, se construye de manera inicial un modelo con intercepto aleatorio. Se parte de la base que se trata de un desenlace binario con una distribución de probabilidad binomial. Se utiliza la función gmler del paquete lme4. Se obtienen los coeficientes y se realiza el cálculo del OR. Los resultados se presentan a continuación. 

\begin{table}[H]
	\caption{Modelo mixto con intercepto aleatorio}
	\label{tab:my-table}
	\begin{tabular}{@{}lcccc@{}}
		\toprule
		& \textbf{Coeficiente} & \textbf{S.E} & \textbf{z} & \textbf{Pr(\textgreater{}|z|)} \\ \midrule
		\textbf{(Intercept)} & 2,1281 & 0,2337 & 9 & \textless{}2e-16 \\
		\textbf{Tiempo} & -1,2978 & 0,1289 & -10 & \textless{}2e-16 \\
		\textbf{OR tiempo} & 0,2731 & IC 95\% & 0.2121 & 0.3516 \\ \bottomrule
	\end{tabular}
\end{table}

El OR para la variable tiempo es de 0.27 con un estrecho intervalo de confianza. Esto es indicativo de un reducción en el chance de consultas completas cuando se incrementa el tiempo en una unidad. Para evaluar la variabilidad explicada por los sujetos y no explicada por el modelo, se realiza el cálculo del coeficiente de correlación intraclase obteniendo un valor de 0.4049.

\subsection{Modelo mixto con intercepto y pendiente aleatorias y cuatro variables independientes como efectos fijos}

Se realiza la construcción de un modelo con intercepto y pendiente aleatorias y las variables tiempo, edad, sexo y medicamentos como variables independientes. Los resultados de los efectos fijos se presentan a continuación. 

\begin{table}[H]
	\caption{Modelo mixto con intercepto y pendiente aleatorias y cuatro variables independientes}
	\label{tab:my-table}
	\begin{tabular}{@{}lcccc@{}}
		\toprule
		& \textbf{Coeficientes} & \textbf{S.E} & \textbf{z} & \textbf{Pr(\textgreater{}|z|)} \\ \midrule
		\textbf{(Intercept)} & 0,78481 & 0,62194 & 1,262 & 0,206999 \\
		\textbf{t} & -1,83762 & 0,26433 & -6,952 & 3,60E-12 \\
		\textbf{edad} & 0,07273 & 0,01936 & 3,757 & 0,000172 \\
		\textbf{sexo} & -0,06839 & 0,28266 & -0,242 & 0,808804 \\
		\textbf{medicamentos} & 0,67957 & 0,29805 & 2,28 & 0,022603 \\ \bottomrule
	\end{tabular}
\end{table}

Para realizar el cálculo de los OR, se realiza la exponenciación de los coeficientes. Los resultados se presentan a continuación. 

\begin{table}[H]
	\centering
	\caption{OR para las variables incluidas}
	\begin{tabular}{rrrr}
		\hline
		& Est & LL & UL \\ 
		\hline
		(Intercept) & 2.19 & 0.65 & 7.42 \\ 
		t & 0.16 & 0.09 & 0.27 \\ 
		edad & 1.08 & 1.04 & 1.12 \\ 
		sexo & 0.93 & 0.54 & 1.63 \\ 
		medicamentos & 1.97 & 1.10 & 3.54 \\ 
		\hline
	\end{tabular}
\end{table}

EL OR de la variable tiempo es de 0.16, indicativo de un reducción en el chance de consultas completas cuando se incrementa el tiempo en una unidad. Para el caso de la variable medicamentos, el chance de realizar las consultas completas es 1.97 en los que tomaron los medicamentos completos en los tres meses con respecto a los que no. La edad tuvo un efecto positivo con un OR de 1.08 y la variable sexo no alcanzó significancia estadística. 

\subsection{Comparación de los modelos}
Se realiza la comparación de los modelos mediante un anova. Los resultados se obtienen a continuación. 

\begin{table}[H]
	\caption{Comparación de los modelos mixtos planteados}
	\label{tab:my-table}
	\begin{tabular}{@{}lcccccccc@{}}
		\toprule
		& \textbf{No. Parámetros} & \textbf{AIC} & \textbf{BIC} & \textbf{logLik} & \textbf{deviance} & \textbf{Chisq} & \textbf{Df} & \textbf{Pr(\textgreater{}Chisq)} \\ \midrule
		\textbf{mix1} & 3 & 512,86 & 525,48 & -253,43 & 506,86 &  &  &  \\
		\textbf{mix4} & 8 & 491,26 & 524,91 & -237,63 & 475,26 & 31,602 & 5 & 7,12E-06 \\ \bottomrule
	\end{tabular}
\end{table}

Se rechaza la hipótesis nula de que los modelos son iguales. El modelo con pendiente e intercepto aleatorio y con cuatro variables se considera el mas adecuado. 

\section{Evalúe la relación entre número de consultas y el resto de variables utilizando un modelo de efectos mixtos.}

Para realizar este punto, se toma en consideración que el desenlace corresponde al número de consultas y es por tanto un desenlace de conteo. La distribución de probabilidad del desenlace es Poisson. Se utiliza la función glmer del paquete lme4. Se inicia con la construcción de un modelo mixto con intercepto aleatorio y sin ninguna otra covariable. Los resultados se presentan a continuación. 

\begin{table}[H]
	\caption{Estimaciones del modelo mixto con intercepto aleatorio}
	\label{tab:my-table}
	\begin{tabular}{@{}lcccc@{}}
		\toprule
		& \textbf{Coeficientes} & \textbf{Error estándar} & \textbf{z} & \textbf{Pr(\textgreater{}|z|)} \\ \midrule
		\textbf{(Intercept)} & 2.37735 & 0,02632 & 90,32 & \textless{}2e-16 \\
		\textbf{t} & -0,33679 & 0,01589 & -21,2 & \textless{}2e-16 \\ \bottomrule
	\end{tabular}
\end{table}

El coeficiente de la variable tiempo es negativo, indicando una reducción en el logaritmo de la tasa de incidencia del número de consultas por cambio en una unidad de tiempo. Se realiza la exponenciación del coeficiente para obtener las razones de tasas de incidencia con los siguientes resultados. 

\begin{table}[H]
	\caption{Razón de tasas de incidencia para el modelo con intercepto aleatorio}
	\label{tab:my-table}
	\begin{tabular}{@{}lccc@{}}
		\toprule
		& \textbf{IRR} & \textbf{IC 95\% LI} & \textbf{IC 95\% LS} \\ \midrule
		\textbf{(Intercept)} & 10,7762692 & 10,2344327 & 11,3467919 \\
		\textbf{tiempo} & 0,7140587 & 0,6921676 & 0,7366421 \\ \bottomrule
	\end{tabular}
\end{table}

La razón de tasas de incidencia del número de consultas se reduce con los periodos de tiempo consecutivos. El IRR tiene un estrecho intervalo de confianza y el coeficiente asociado fue estadísticamente significativo. 

A partir de lo encontrado en los modelos anteriores, se contruye un modelo mixto con intercepto y pendiente aleatoria, introduciendo como variables el tiempo, el consumo de medicamentos y la edad. Cuando se adiciona la variable edad, el modelo no converge. Se introduce como término cuadrático y como interacción pero no se logra convergencia. Los resultados se presentan a continuación. 

\begin{table}[H]
	\caption{Modelo mixto con intercepto y pendiente aleatoria y tres variables independientes.}
	\label{tab:my-table}
	\begin{tabular}{@{}lcccc@{}}
		\toprule
		& \textbf{Coeficiente} & \textbf{Error estándar} & \textbf{z} & \textbf{Pr(\textgreater{}|z|)} \\ \midrule
		\textbf{(Intercept)} & 2,37337 & 0,03031 & 78,302 & \textless 2,00E-16 \\
		\textbf{Tiempo} & -0,35464 & 0,01985 & -17,863 & \textless 2,00E-16 \\
		\textbf{Sexo} & -0,05995 & 0,04375 & -1,37 & 0,17056 \\
		\textbf{Medicamentos} & 0,12684 & 0,0418 & 3,034 & 0,00241 \\ \bottomrule
	\end{tabular}
\end{table}

Para los efectos aleatorios de la variable tiempo la correlación fue de -0.66. Esto coincide con el coeficiente observado con signo negativo. La variable tiempo y medicamentos son estadísticamente significativas. 
Para obtener los IRR se realiza la exponenciación de los coeficientes. Los resultados se presentan a continuación. 

\begin{table}[H]
	\centering
	\begin{tabular}{rrrr}
		\hline
		& IRR & IC 95\% LI & IC 95\% LS \\ 
		\hline
		(Intercept) & 10.73 & 10.11 & 11.39 \\ 
		t & 0.70 & 0.67 & 0.73 \\ 
		sexo & 0.94 & 0.86 & 1.03 \\ 
		medicamentos & 1.14 & 1.05 & 1.23 \\ 
		\hline
	\end{tabular}
\end{table}

Las IRR para el número de consultas se reducen en el tiempo y se incrementan en los pacientes que toman los medicamentos completos. La variable sexo no fue significativa. 

Finalmente se realiza la comparación de los modelos mediante un anova. Los resultados se presentan a continuación. 

\begin{table}[H]
	\caption{ANOVA comparando los dos modelos presentados }
	\label{tab:my-table}
	\begin{tabular}{@{}lcccccccc@{}}
		\toprule
		& \textbf{Parámetros} & \textbf{AIC} & \textbf{BIC} & \textbf{logLik} & \textbf{deviance} & \textbf{Chisq} & \textbf{Gl} & \textbf{Pr(\textgreater{}Chisq)} \\ \midrule
		\textbf{Intercepto aleatorio} & 3 & 2491,4 & 2504 & -1242,7 & 2485,4 &  &  &  \\
		\textbf{Modelo completo} & 7 & 2473 & 2502,4 & -1229,5 & 2459 & 26,439 & 4 & 2,58E-05 \\ \bottomrule
	\end{tabular}
\end{table}

Los resultados favorecen el modelo completo que incluye pendiente e intercepto aleatorio y las demás variables como efectos fijos. 


\end{document}
